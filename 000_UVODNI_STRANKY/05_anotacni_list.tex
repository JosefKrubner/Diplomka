\thispagestyle{empty}
\section*{Anotační list}
\def\arraystretch{2}%  1 is the default, change whatever you need
\begin{table}[ht!]
    \centering
    \begin{tabular}{p{.21\textwidth}|p{.73\textwidth}}
    Název práce: & Zlepšení termodynamických vlastností vysokorychlostní DRTA sondy pomocí numerických simulací \\ \hline
    Title: & Improvement of thermodynamic properties of a high-speed DRTA probe by numerical simulations \\ \hline
    Autor: & Bc. Josef Krubner \\ \hline
    Studijní program: & Aplikované vědy ve strojním inženýrství \\ \hline
    Druh práce: & Diplomová \\ \hline
    Vedoucí práce & Ing. Michal Schmirler, Ph.D. \\ \hline
    Konzultant & doc. Ing. Jan Halama, Ph.D. \\ \hline
    Abstrakt: & Práce se zaměřuje na vývoj DRTA sondy pro měření vysokých podzvukových rychlostí plynů, která je založena na principu měření rovnovážných teplot pomocí odporových teplotních snímačů s rozdílnými restitučními faktory. Zkoumán byl vliv dílčích konstrukčních úprav prototypu na jeho termodynamické vlastnosti s využitím CFD simulací. Z nabytých poznatků byla navržena nová geometrie sondy, jejíž chování bylo taktéž simulováno. \\ \hline
    Abstract: & This work focuses on the design of high subsonic gas speed DRTA probe, which is based on the priciples of recovery temperature measurement using two RTD sensors with different recovery factors. Using CFD simulations, analysis of constructional changes' influence on the probe's thermodynamic properties were made. With the use of acquired knowledge new probe's prototype geometry was presented and analysed as well.  \\ \hline
    Klíčová slova: & návrh sondy pro měření rychlosti, měření rychlosti plynů, podzvukové proudění, restituční faktor, rovnovážná teplota, CFD simulace\\ \hline
    Keywords: & velocimetry probe design, gas velocimetry, subsonic flow, recovery factor, recovery temperature, CFD simulation
    \end{tabular}
\end{table}
\def\arraystretch{1}%  1 is the default, change whatever you need