\section*{Úvod}
\addcontentsline{toc}{section}{Úvod}

    Měření parametrů proudění plynů je běžnou součástí průmyslových či výzkumných procesů a konkrétně stanovení rychlosti je samo o sobě velice širokou vědní disciplínou, k níž lze přistupovat mnoha způsoby v závislosti na charakteru proudění a proudícího média, či na povaze měřené rychlosti – v praxi se lze setkat s měřením okamžité rychlosti, rychlostních profilů, či celých $2D/3D$ rychlostních polí pomocí optických metod. 
    
    Tato práce je zaměřena na vývoj anemometrické sondy využívající dvojici odporových teplotních snímačů, která je určena pro oblast vysokých podzvukových rychlostí. Jedná se o poměrně neprobádaný pohled na tuto problematiku, který by mohl konkurovat v praxi již ověřeným metodám, mezi které patří například Prandtlova sonda.

    Hlavním cílem práce bylo zlepšení termodynamických vlastností sondy prostřednictvím vybraných konstrukčních úprav. Vliv dílčích změn geometrie byl zkoumán s využitím numerických simulací v prostředí software Ansys Fluent. Tento přístup byl zvolen s ohledem na předpokládáné vysoké množství zkoumaných konstrukčních úprav, což by jinak bylo při experimentálním testování problematické jak časově, tak i s ohledem na nutnost výroby jednotlivých prototypů.

    \vspace{\baselineskip}

    Práce je členěna na celkem pět kapitol. První obsahuje úvod do metodiky měření teplot při vysokých podzvukových rychlostech. S ohledem na zkoumanou problematiku je zaměřena hlavně na využití teplotních snímačů. Druhá kapitola popisuje princip fungování zkoumané DRTA sondy a seznamuje čtenáře s výchozí geometrií, se kterou je v dalších kapitolách pracováno. Třetí kapitola představuje použitý výpočetní model včetně uvedení fyzikálních rovnic a popisu výpočetní oblasti a numerického řešiče. Jejím cílem je umožnit v budoucnu realizaci dalších simulací konstrukčních úprav za stejných podmínek, za jakých byly prováděny v této práci. Čtvrtá, nejrozsáhlejší, kapitola se zabývá analýzou a zhodnocením vlivu dílčích konstrukčních úprav modelu sondy na jeho termodynamické vlastnosti. V páté kapitole je představen finální návrh modelu DRTA sondy spolu s analýzou termodynamických vlastností. 



    