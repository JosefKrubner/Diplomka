\section{CFD model}
    Během výpočtů byl použit předpoklad stacionárního vazkého proudění ideálního plynu, od čehož se odvíjela i forma níže uvedených rovnic. 
    \subsection{Základní systém rovnic}
        
        \subsubsection{Rovnice kontinuity}
            Zákon zachování hmotnosti je pro stlačitelné stacionární proudění popsán následující rovnicí:
            \begin{equation} \label{eq:rovnice-kontinuity}
                \nabla \cdot  \brac{\rho \Vec{u}} = 0  
            \end{equation}
            \noindent kde $\rho \unit{\frac{kg}{m^3}}$ je hustota a $\Vec{u} \unit{\frac{m}{s^2}}$  je rychlost proudění.
        
        \subsubsection{Pohybová rovnice}
            Přenos hybnosti je popsán Navier-Stokesovými rovnicemi pro stacionární proudění:
            \begin{equation} \label{eq:pohybova-rovnice}
                \nabla \cdot \brac{\rho \Vec{u} \otimes \Vec{u}} = - \nabla p + \nabla \cdot \Tensor{\tau} + \Vec{f}
            \end{equation}
            \noindent kde $p \unit{Pa}$ je statický tlak, $\Vec{f} \unit{\frac{N}{m^3}}$ je vektor vnějších sil a $\Tensor{\tau} \unit{\frac{N}{m^2}}$ je tenzor vazkých napětí daný následujícím vztahem:
            \begin{equation} \label{eq:tenzor-vazkych-napeti}
                \Tensor{\tau} = \mu \Brac{\nabla \Vec{u} + \nabla \Vec{u}^T - \frac{2}{3} \brac{\nabla \cdot \Vec{u}} I}
            \end{equation}
            \noindent kde $\mu \unit{Pa \cdot s}$ je dynamická viskozita a $I \unit{1}$ je jednotková matice.
            
        \subsubsection{Energetická rovnice}
            Řešení stlačitelného proudění vyžaduje doplnění energetické rovnice, kterou lze zapsat následovně:
            \begin{equation} \label{eq:energeticka-rovnice}
                \nabla \cdot \brac{\rho \Vec{u} H + \Vec{q} - \Tensor{\tau} \cdot \Vec{u}} = 0
            \end{equation}
            \noindent kde $H \unit{\frac{J}{kg}}$ je celková měrná entalpie a $\Vec{q} \unit{\frac{W}{m^2}}$ je vektor tepelného toku.
        \subsubsection{Konstitutivní vztahy}
        \smallSection{Stavová rovnice ideálního plynu}
        Rovnice popisuje vazbu mezi stavovými veličinami tekutiny:
        \begin{equation} \label{eq:stavova-rovnice}
            \frac{p}{\rho} = r T
        \end{equation}
        \noindent kde $T \unit{K}$ je termodynamická teplota a $r \unit{\frac{J}{kg \cdot K}}$ je měrná plynová konstanta, pro vzduch rovna $287.2 \unit{\frac{J}{kg \cdot K}}$.
        
        \smallSection{Celková měrná entalpie}
        Měrnou entalpii proudění $h \unit{\frac{J}{kg}}$ lze určit ze vztahu:
        \begin{equation} \label{eq:entalpie}
            h = c_p T = e + \frac{p}{\rho} = c_v T + \frac{p}{\rho}
        \end{equation}
        \noindent kde $c_p \, , \; c_v  \unit{\frac{J}{kg \cdot K}}$ jsou měrné tepelné kapacity za konstantního tlaku, resp. konstantního objemu a $e \unit{\frac{J}{kg}}$ je měrná energie. Přičtením měrné kinetické energie proudění dostáváme celkovou měrnou entalpii $H$:
        \begin{equation} \label{eq:celkova-entalpie}
            H = h + \frac{\norm{\Vec{u}}^2}{2}
        \end{equation}
        
    \subsection{Model turbulence}
        
        Transportní rovnice pro turbulentní kinetickou energii má následující tvar:
        \begin{equation} \label{eq:rovnice-pro-k} 
            \nabla \cdot \brac{\rho k \Vec{u}} = \nabla \cdot \brac{\Gamma _k \nabla k} + G _k - Y _k
        \end{equation}
        \noindent kde dílčí členy jsou popsány následovně:
        \begin{multicols}{2}
            \vztah{Produkce $k$}{G_k = \mu _t S^2}

            \vztah{Disipace $k$}{Y _k = \rho \beta ^* _\infty k \omega}

            \vztah{Turbulentní vazkost}{\mu _t = \frac{\rho k}{\omega} \frac{1}{\max \brac{\frac{1}{\alpha ^*}, \frac{S F_2}{a_1 \omega}}}}
            
            \vztah{Difuzivita $k$}{\Gamma _k = \mu + \frac{\mu _t}{\sigma _k}}

            \vztah{Turbulentní Reynoldsovo číslo}{Re_t = \frac{\rho k}{\mu \omega}}

            \vztah{Turbulentní Prandtlovo číslo pro k}{\sigma _k = \frac{1}{\frac{F_1}{\sigma _{k,1}}+\frac{1-F_1}{\sigma _{k,2}}}}

            \vztah{Difuze $\omega$}{D _\omega ^+ = \max \Brac{2 \rho \frac{1}{\sigma _{\omega , 2}}\frac{1}{\omega} \nabla k \cdot \nabla \omega, 10^{-10}}}

            \vztah{Tenzor rychlosti deformace}{\Tensor{S} = \frac{1}{2} \brac{\nabla \Vec{u} + \nabla \Vec{u} ^T}}

            \vzorec{S = \sqrt{2 \Tensor{S} : \Tensor{S}}}
            
            \vzorec{F_2 = \tanh \brac{\Phi _2 ^2}}

            \vzorec{\Phi _2 = \max \brac{ 2 \frac{\sqrt{k}}{0.09 \omega y}, \frac{500 \mu}{\rho y^2 \omega}}}

            \vzorec{F_1 = \tanh \brac{\Phi _1 ^4}}

            \vzorec{\Phi _1 = \min \Brac{\max \brac{\frac{\sqrt{k}}{0.09 \omega y},\frac{500 \mu}{\rho y^2 \omega}}, \frac{4 \rho k}{\sigma _{\omega,2} D _\omega ^+ y^2}}}

            \vzorec{\alpha ^* = \alpha ^* _\infty \brac{\frac{\alpha _0 ^* + \frac{Re_t}{R_k}}{1 + \frac{Re_t}{R_k}}}}

            \vzorec{\alpha ^* _0 = \frac{\beta _i}{3}}

            \vzorec{\beta _i = F_1 \beta _{i,1} + \brac{1 - F_1} \beta _{i,2}}
        \end{multicols}

        Obdobně lze zapsat rovnici pro specifickou rychlost disipace:
        \begin{equation} \label{eq:rovnice-pro-omega}
            \nabla \cdot \brac{\rho \omega \Vec{u}} = \nabla \cdot \brac{\Gamma _\omega \nabla \omega} + G _\omega - Y _\omega
        \end{equation}
        \noindent která se skládá z následujících členů:
        \begin{multicols}{2}
            \vztah{Produkce $\omega$}{G _\omega = \frac{\alpha \alpha ^*}{\nu _t} G _k}

            \vztah{Disipace $\omega$}{Y _\omega = \rho \beta_i \omega ^2}
            
            \vztah{Difuzivita $\omega$}{\Gamma _\omega = \mu + \frac{\mu _t}{\sigma _\omega}}

            \vztah{Turbulentní Prandtlovo číslo pro $\omega$}{\sigma _\omega = \frac{1}{\frac{F_1}{\sigma _{\omega,1}}+\frac{1-F_1}{\sigma _{\omega,2}}}}
            
            \vzorec{\alpha = \frac{\alpha _\infty}{\alpha ^*}\brac{\frac{\alpha _0 + \frac{Re_t}{R_\omega}}{1 + \frac{Re_t}{R_\omega}}}}

            \vzorec{\alpha _\infty = F_1 \alpha _{\infty , 1} + \brac{1 - F_1} \alpha _{\infty , 2}}

            \vzorec{\alpha _{\infty , 1} = \frac{\beta _{i,1}}{\beta _\infty ^*} - \frac{\chi ^2}{\sigma _{\omega , 1} \sqrt{\beta _\infty ^*}}}

            \vzorec{\alpha _{\infty , 2} = \frac{\beta _{i,2}}{\beta _\infty ^*} - \frac{\chi ^2}{\sigma _{\omega , 2} \sqrt{\beta _\infty ^*}}}
        \end{multicols}

        \begin{table}[ht!]
            \centering
            \begin{tabular}{l|l|l|l|l|l|l|l}
            $\alpha _\infty ^*$ &  $\alpha _0$    & $\beta _\infty ^*$   & $R_k$                & $R_\omega$       & $\chi$           & $a_1$    \\ \hline
            $0.41$              &  $\frac{1}{9}$  & $0.09$               & $6$                  & $2.95$           & $0.25$           & $0.31$   \\ \hline
            $\sigma _{k,1}$     & $\sigma _{k,2}$ & $\sigma _{\omega,1}$ & $\sigma _{\omega,2}$ & $\beta _{i,1}$   & $\beta _{i,2}$   &          \\ \hline
            $1.176$             & $1$             & $2$                  & $1.168$              & $0.075$          & $0.0828$         &
            \end{tabular}
            \caption{Konstanty modelu k-$\omega$ SST.}
            \label{tab:konstanty-turbulence}
            \end{table}
    \newpage
    \subsection{Výpočetní geometrie}
        Vzhledem k charakteru řešeného problému byla geometrie proměnlivá. Jednotícím prvkem byla přítomnost alespoň jednoho ze dvou teplotních čidel, jehož restituční faktor byl zkoumán. Podle aktuální simulace se však měnilo uspořádání a přítomnost dalších geometrických prvků, jako například stínění.
    
        \subsubsection{Výpočetní oblast}
            Výpočty byly prováděny na geometrii umístěné v kontrolní oblasti tvaru válce o průměru $250 \Unit{mm}$ a délce $450 \Unit{mm}$. Vzhledem k rozměrům čidel, respektive celkové konstrukce, se jednalo o dostatečně velký kontrolní objem, který neměl ovlivňovat proudění okolo sondy. Veškeré měřené geometrie byly ve válci umístěné $100 \Unit{mm}$ od vstupní oblasti, viz obrázek \ref{fig:vypocetni-oblast}, ze kterého je patrné i umístění souřadného systému, na který bude dále v práci odkazováno.
            
            \begin{figure}[ht!]
                \centering
                \includegraphics[width=\textwidth]{300_VYPOCETNI_MODEL/Vypocetni_oblast.png}
                \caption{Výpočetní oblast s vyznačením souřadného systému a polohy měřených geometrií.}
                \label{fig:vypocetni-oblast}
            \end{figure}
        
        \subsubsection{Využití symetrie}
            U všech zkoumaných geometrií se nacházela alespoň jedna rovina symetrie – bylo tedy možné využít této výhody pro úsporu výpočetního výkonu. Veškeré simulace uvedené v kapitolách \ref{sec:konstrukcni-upravy} a \ref{sec:finalni-geometrie} s výjimkou analýzy směrové citlivosti v rovině $XZ$ byly provedeny s využitím symetrie výpočetního modelu, viz obrázek \ref{fig:vypocetni-oblast-symetrie}.
            
            \begin{figure}[ht!]
                \centering
                \includegraphics[width=\textwidth]{300_VYPOCETNI_MODEL/Vypocetni_oblast_symetrie.png}
                \caption{Výpočetní oblast pro řešení symetrických úloh.}
                \label{fig:vypocetni-oblast-symetrie}
            \end{figure}
         
	\newpage
        \subsubsection{Materiály}
            Během výpočtů byly uvažovány celkem tři materiály, ze kterých se skládala geometrie – trubice byla tvořena mosazí, čidla byla uvažována jako homogenní tělesa z keramiky $Al_2 O_3$ a těsnění bylo reprezentováno pryží. Použité fyzikální vlastnosti jednotlivých materiálů jsou uvedeny v tabulce \ref{tab:materialy}.
            
            \renewcommand{\arraystretch}{2}
            \begin{table}[ht!]
                \centering
                %\resizebox{.7\textwidth}{!}{%
                \begin{tabular}{l|l|l|l}
                                                                                  & Mosaz                        & Pryž                        &  Keramika                      \\ \hline
                    Hustota $\unit{\frac{kg}{m^3}}$                               & \multicolumn{1}{c|}{$8730$}  & \multicolumn{1}{c|}{$1100$} & \multicolumn{1}{c}{$3500$}     \\ \hline
                    Měrná tepelná kapacita $\unit{\frac{J}{kg \cdot K}}$          & \multicolumn{1}{c|}{$400$}   & \multicolumn{1}{c|}{$1300$} & \multicolumn{1}{c}{$700$}      \\ \hline
                    Tepelná vodivost $\unit{\frac{W}{m \cdot K}}$                 & \multicolumn{1}{c|}{$96$}    & \multicolumn{1}{c|}{$0.09$} & \multicolumn{1}{c}{$30$}      
                \end{tabular}%
                %}
                \caption{Fyzikální vlastnosti použitých materiálů.}
                \label{tab:materialy}
            \end{table}
            
            Jako proudící médium byl uvažován vzduch splňující stavovou rovnici ideální plynu (viz vztah \ref{eq:stavova-rovnice}) s následujícími vlastnostmi:

		\begin{table}[ht!]
			\centering
			%\resizebox{.7\textwidth}{!}{%
			\begin{tabular}{l|l}
				Měrná plynová konstanta $\unit{\frac{J}{kg \cdot K}}$ & Poissonovo číslo $\unit{1}$ \\ \hline
				\multicolumn{1}{c|}{$287$}                                                                        & \multicolumn{1}{c}{$1.4$}                \\ \hline
				Tepelná vodivost $\unit{\frac{W}{m \cdot K}}$ & Dynamická viskozita $\unit{Pa \cdot s}$ \\ \hline				
				\multicolumn{1}{c|}{$2.42 \cdot 10^{-2}$}                          &  \multicolumn{1}{c}{$1.7894 \cdot 10^{-05}$}     
			\end{tabular}%
			%}
			\caption{Fyzikální vlastnosti vzduchu.}
			\label{tab:vzduch}
		\end{table}
            
            
     \newpage      
    \subsection{Okrajové podmínky}
        \subsubsection{Hranice výpočetní oblasti}
			Hranice válcové kontrolní oblasti byla rozdělena na tři části s odlišnými okrajovými podmínkami – podstavy válce představovaly vstup a výstup a jeho plášť poté nenarušený proud (viz obrázek \ref{fig:okrajove-podminky}). Ve všech oblastech byly předepsány hodnoty uvedené v tabulce \ref{tab:spolecne-op}. Ve vstupní oblasti byla dále zadávána rychlost proudění, respektive velikost vektoru rychlosti a jeho směrové cosiny (využito při analýze směrové citlivosti). Hranice nenarušeného proudu měla předepisovánu hodnotu Machova čísla a směr proudění (opět ve formě směrových cosinů vektoru rychlosti). 

	     \begin{figure}[ht!]
                    \centering
                        \begin{subfigure}{0.3\textwidth}
                             \centering
                             \captionsetup{width=.9\linewidth}
                                \includegraphics[width=\textwidth]{300_VYPOCETNI_MODEL/op-inlet.png}
                             \caption{Vstupní oblast.}
                             \label{fig:inlet}
                         \end{subfigure}
                         \begin{subfigure}{0.3\textwidth}
                             \centering
                             \captionsetup{width=.9\linewidth}
                             \includegraphics[width=\textwidth]{300_VYPOCETNI_MODEL/op-farfield.png}
                             \caption{Nenarušený proud.}
                             \label{fig:farfield}
                         \end{subfigure}
					\begin{subfigure}{0.3\textwidth}
                             \centering
                             \captionsetup{width=.9\linewidth}
                             \includegraphics[width=\textwidth]{300_VYPOCETNI_MODEL/op-outlet.png}
                             \caption{Výstupní oblast.}
                             \label{fig:outlet}
                         \end{subfigure}
                    \caption{Části hranice pro aplikování okrajových podmínek (v jednotlivých obrázcích označeny žlutou barvou).}
                    \label{fig:okrajove-podminky}
         \end{figure}

         \begin{table}[ht!]
            \centering
            %\resizebox{.7\textwidth}{!}$ & Směšovací délka $\unit{m}$  \\ \hline
                \multicolumn{1}{c|}{2.5}         & \multicolumn{1}{c}{0.01}   
            \end{tabular}%
            %}
            \caption{Hodnoty předepisované na hranici kontrolní oblasti.}
            \label{tab:spolecne-op}
        \end{table}

         Výchozí rychlostí použitou pro testování bylo $250 \Unit{\frac{m}{s}}$, tomu odpovídá při teplotě $300 \Unit{K}$ Machovo číslo $0.72$. Nebude-li dále uvedeno jinak, pak byly pro výpočet použity právě tyto hodnoty.
			
        
        \subsubsection{Stěny}
            Při numerických simulacích bylo pro vyhodnocení restitučních faktorů třeba počítat s přestupem tepla do pevných látek a s jeho šířením objemem. V místech kontaktu proudícího média se stěnami geometrie byla proto použita podmínka sdílené teploty – teplota na hranici tekutiny byla přenesena na hranici tělesa.

    \newpage
    \subsection{Výpočetní síť} \label{sec:vypocetni-sit}
        Vytváření modelů probíhalo v prostředí software Autodesk Inventor (verze 2021 a 2022), odkud byly následně vyexportovány ve formátu \textit{.dwg}. K přípravě pro síťování byl následně použit software Ansys SpaceClaim (verze 2020b-2021b), jehož účel spočíval primárně ve sdílení topologie modelu, vytváření jmenných sekcí a exportu do optimalizovaného formátu \textit{.pmdb}. Samotné síťování poté probíhalo v software Ansys Fluent (verze 2020b-2021b).
        \subsubsection{Povrchová síť}
            Prvním krokem při vytváření výpočetní sítě pro řešič bylo importování geometrie (soubor \textit{.pmdb}) a vysíťování jejích ploch pomocí triangulace. Zde bylo použito následující nastavení:

            \begin{table}[ht!]
                \centering
                \begin{tabular}{l|l}
                    $\frac{\textrm{Minimální}}{\textrm{Maximální}}$ velikost elementů $\unit{mm}$                                            & Poměrný růst velikosti elementů $\unit{1}$ \\ \hline
                    \multicolumn{1}{c|}{$\frac{0.1}{15}$}                                                              & \multicolumn{1}{c}{$1.2$}                  \\ \hline
                    Maximální úhel překlenutí $\unit{deg}$                                                         & Minimální dělení hran $\unit{1}$           \\ \hline
                    \multicolumn{1}{c|}{$10$}                                                                      & \multicolumn{1}{c}{$3$}                     
                \end{tabular}
                \caption{Předepisované hodnoty při vytváření povrchové sítě.}
                \label{tab:povrchova-sit-nastaveni}
            \end{table}

            Kvalita povrchové sítě byla následně kontrolována, aby šikmost žádného elementu nepřesáhla $0.5$. Šikmost představuje odchylku geometrie buňky od optimálního tvaru (v případě triangulace se jedná o rovnostranný trojúhelník). Její hodnota se pohybuje mezi $0 \div 1$, kde $0$ odpovídá nejlepší kvalitě. Pro správný průběh a konvergenci výpočtů je doporučeno, aby maximální šikmost nepřesahovala $0.95$ a aby se průměrná šikmost pohybovala nejvýše okolo hodnoty $0.33$ \cite{Ansys2020User}. Tato doporučení platí pro konečnou objemovou síť, která se používá během výpočtů, nicméně počáteční kvalita povrchové sítě má zásadní vliv na jakost následujícího síťování.
        \subsubsection{Zjemnění v mezní vrstvě}
            Pro dosažení přijatelné přesnosti výpočtu přestupu tepla ze vzduchu do těles bylo třeba vytvořit dostatečně jemnou síť v oblasti mezní vrstvy. K tomu byly využity prismatické buňky v místech kontaktu tekutiny s měřenou geometrií. Cílem bylo dosažení průměrné bezrozměrné vzdálenosti od stěny co nejblíže $1$. Toho bylo docíleno pomocí natavení uvedeného v tabulce \ref{tab:mezni-vrstva-nastaveni}. Příklad rozložení $y_+$ u teplotního čidla je uveden na obrázku \ref{fig:yplus-stineni-A}.

            \begin{table}[ht!]
                \centering
                \begin{tabular}{ll}
                    \multicolumn{2}{l}{Míra natažení prvního elementu (aspect ratio) $\unit{1}$}                            \\ \hline
                    \multicolumn{2}{c}{$6.2$}                                                                        \\ \hline
                    \multicolumn{1}{l|}{Poměrný růst velikosti elementů $\unit{1}$} & Počet prismatických vrstev $\unit{1}$ \\ \hline
                    \multicolumn{1}{c|}{$1.2$}                                      & \multicolumn{1}{c}{$10$}                
                \end{tabular}
                \caption{Předepisované hodnoty při vytváření prismatických buněk.}
                \label{tab:mezni-vrstva-nastaveni}
            \end{table}

            \begin{figure}[ht!]
                \centering
                \includegraphics*[width=\textwidth  ]{300_VYPOCETNI_MODEL/yplus-stineni-A.eps}
                \caption{Graf četností hodnot bezrozměrné vzdálenosti od stěny čidla A pro úlohu z kapitoly \ref{sec:stineni-A}. Průměrná hodnota je pro tento případ rovna $0.945$.}
                \label{fig:yplus-stineni-A}
            \end{figure}

            \newpage


        \subsubsection{Objemová síť}

        Při vytváření objemové sítě byly zvoleny polyhedrální buňky, které umožňují dosahovat přesnějších řešení oproti starším typům elementů při shodných počtech buňek. Umožňují navíc lepší odhad gradientu díky vyššímu počtu stěn a obecně lze s jejich použitím dosahovat lepší kvality sítě \cite{Sosnowski2018}. Postup generování objemové sítě byl následující:

        \begin{enumerate}
            \item Konverze povrchové triangulace na polygonální síť
            \item Vygenerování vrstev prismatických buněk
            \item Iterační generování polyhedrální objemové sítě ve zbytku objemu
        \end{enumerate}

        Maximální šikmost hotové sítě se vždy pohybovala pod hodnotou $0.85$. Počty buňek se pohybovaly v rozmezí $450 \div 550$ tisíc pro symetrické úlohy a $850 \div 950$ tisíc pro úlohy bez využití symetrie. Příklad výpočetní sítě je uveden na obrázku \ref{fig:sit-detail-stineni-A}.

        \begin{figure}
            \centering
            \includegraphics[width=\textwidth]{300_VYPOCETNI_MODEL/mesh_sym_odvetrani_A.png}
            \caption{Pohled na výpočetní síť z kapitoly \ref{sec:stineni-A} ze strany symetrie.}
            \label{fig:sit-detail-stineni-A}
        \end{figure}
        
    \subsection{Numerický řešič}
        \subsubsection{Metoda konečných objemů}
        \subsubsection{Odhad gradientu}
        \subsubsection{Aproximace hodnot na stěnách}
        \subsubsection{Numerické schéma}
        \subsubsection{Určení restitučních faktorů}

        
        