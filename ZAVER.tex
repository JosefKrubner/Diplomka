\section*{Závěr}
\addcontentsline{toc}{section}{Závěr}

V rámci práce byla navržena nová geometrie DRTA sondy, která slouží k měření vysokých podzvukových rychlostí. Návrhu předcházela numerická analýza vlivu dílčích konstrukčních úprav na termodynamické vlastnosti sondy, konkrétně na restituční faktory jejích teplotních čidel.

První kapitola byla věnována problematice dynamického ohřevu spojeného s měřením teplot při vysokých podzvukových rychlostech. Byl zde popsán restituční faktor a s ním spojená rovnovážná teplota. S využitím těchto znalostí byl představen způsob určení stagnační teploty pomocí teplotních snímačů a dále bylo popsáno nepřímé měření statické teploty.

Ve druhé kapitole byl představen princip metody DRTA pro měření vysokých podzvukových rychlostí a byla popsána výchozí konstrukce prototypu sondy. V závěru kapitoly byly uvedeny cíle numerických simulací.

Cílem třetí kapitoly byl popis použitého výpočetního modelu. Byly představeny příslušné fyzikální rovnice a model turbulence. Následoval popis výpočetní geometrie a použitých materiálů, následovaný představením aplikovaných okrajových podmínek. Zbývající část této kapitoly byla věnována přípravě výpočtu – byl zde popsán postup tvorby výpočetní sítě a nastavení výpočtu.

Čtvrtá kapitola se věnovala analýze dílčích konstrukčních úprav, kterým předcházela studie vlivu jemnosti sítě na výsledky výpočtu, na základě čehož bylo zvoleno nastavení parametrů povrchové sítě, které bylo dále aplikováno ve všech dalších simulacích. 

V poslední kapitole byla navržena nová konstrukce DRTA sondy. Vedle změn rozměrů a rozložení jednotlivých geometrických prvků konstrukce bylo dále přidáno stínění k čidlu B. Následně byla tato upravená geometrie zkoumána obdobně, jako dílčí konstrukční úpravy. Bylo zkoumáno chování restitučních faktorů při změně rychlosti proudění a při změně natočení sondy. Dále bylo simulováno chování pro různé materiály sondy. Zde se ukázal jako nejvhodější polykarbonát, zejména díky své nízké tepelné vodivosti. 

Úprava geometrie měla největší přínos ohledně směrové necitlivosti sondy. Toho bylo dosaženo především díky přidanému stínění u čidla B. Zlepšení bylo patrné také při změně rychlostí proudění, kdy došlo k částečnému vyrovnání průběhu rozdílu restitučních faktorů sondy. 

Jmenovitá hodnota rozdílu restitučních faktorů byla stanovena jako $0.1232$, což odpovídá rychlosti proudění $250 \Unit{\frac{m}{s}}$. Chyba měření rychlosti při uvažování této hodnoty se ukázala ve většině případů (s výjimkou velkých natočení v rovině symetrie sondy a rychlostí překračujících $300 \Unit{\frac{m}{s}}$) menší, než $2.5 \Unit{\%}$.

Výstupy této práce by bylo vhodné experimentálně ověřit – dalšími navazujícími kroky by tak mohla být výroba prototypu upravené DRTA sondy a její následné testování. 